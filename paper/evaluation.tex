\section{Evaluation}

\subsection{Performance Evaluation}


We compare \NM{} with multiple popular allocators, such as Linux's glibc allocator, TcMalloc~\cite{tcmalloc}, or NUMA Aware TcMalloc~\cite{tcmallocnew}, jemalloc~\cite{jemalloc}, Scalloc~\cite{Scalloc}. For the simplicity, NUMA aware TcMalloc is called as TcMalloc-NUMA in the remainder of this paper. 

Performance evaluation will be performed on two different hardware as further described in the Table~\ref{table:Machine}:

%numactl --hardware Finding out latency information across nodes. 
% numactl --show Will show the physical core id information. 

\begin{table}[h]
  \footnotesize
  \setlength{\tabcolsep}{1.0em}
\begin{tabular}{c c c}
\hline
System & \textbf{Machine A} & \textbf{Machine B} \\ \hline
CPUs/Model & Xeon Gold 6138	& \\ \hline
NUMA Nodes & 2 & 8 \\ \hline
Physical Cores & 2$\times$20 & 8$\times$16 \\ \hline
Node Latency & \specialcell{local: 1.0 \\ 1 hop: 2.1} & \\ \hline
Interconnect Bandwidth & 8GT/s & \\ \hline
Memory Bandwidth & 19.87 GB/s & \\ \hline
  \end{tabular}
  \centering
  \caption{Machine Specifications.\label{table:Machine}}
\end{table}



%Scalloc is located in https://github.com/cksystemsgroup/scalloc

% Christoph also has a benchmark suite to evaluate the performance of memory allocator. 

% export LD_LIBRARY_PATH="/usr/local/lib/" for running tcmalloc.


\paragraph{Application Statistics} We also checked the corresponding details of 