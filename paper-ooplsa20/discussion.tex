\section{Limitation}
\label{sec:limit}

This section describes some limitations of \NM{}. 

The first limitation of \NM{} is that it consumes more memory than some popular allocators. Based on our analysis, the memory consumption is caused by \NM{}'s bag mechanism. Currently, \NM{} employs one MB as a bag, which indicates that objects of each size will occupy at least of 1 MB, when transparent huge page support is enabled. \NM{}'s design achieves a fast lookup on the metadata, but will utilize more memory unfortunately. We will investigate whether reducing the size of a bag could help reduce the memory consumption in the future.

The second limitation is that \NM{} will crash less often by preallocating a huge chunk of memory from the OS in the beginning. If an invalid reference is landed within a pre-allocated range, a program will not crash, different from other allocators. Instead, \NM{} aims to achieve the high performance over the reliability. Therefore, we believe that this limitation is acceptable.  