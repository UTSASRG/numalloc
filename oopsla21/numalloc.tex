\documentclass[10pt,journal,compsoc]{IEEEtran}

\usepackage{graphics} % for EPS, load graphicx instead
%\usepackage[T1]{fontenc}
\usepackage{url}
\usepackage{hyperref}
\usepackage{txfonts}
\usepackage{pslatex}    
\usepackage{pifont}
%\usepackage{bbding}
\usepackage{multirow}
\usepackage{makecell}
\usepackage[justification=centering]{caption}
\usepackage{xspace}
\usepackage{comment}
\usepackage{listings}
\usepackage{tikz}
\usepackage{calc}
%\usepackage{fancyvrb}
\usepackage{xcolor}
\usepackage[keeplastbox]{flushend}

\ifCLASSOPTIONcompsoc
  % The IEEE Computer Society needs nocompress option
  % requires cite.sty v4.0 or later (November 2003)
  \usepackage[nocompress]{cite}
\else
  % normal IEEE
  \usepackage{cite}
\fi
\newcommand{\cmark}{\ding{51}}%
\newcommand{\xmark}{\ding{55}}%

\newcommand{\todo}[1]{{\color{red}\bfseries [[#1]]}}
\newcommand{\TP}[1]{{\color{red}\bfseries [[#1]]}}

\newcommand{\Watcher}{{Watcher}}
\newcommand{\WA}{{Watcher}}
\newcommand{\ASAN}{{ASan}}
\newcommand{\DT}{{DoubleTake}}

\newcommand{\OB}{\texttt{OpenBSD}}
\newcommand{\DieHarder}{\texttt{DieHarder}}
\newcommand{\DL}{\texttt{DLmalloc}}
\newcommand{\JE}{\texttt{jemalloc}}
\newcommand{\NA}{\texttt{NUMAlloc}}
\newcommand{\NM}{\texttt{NUMAlloc}}
\newcommand{\TN}{TcMalloc-NUMA}

\newcommand{\pthread}{\texttt{pthread}}
\newcommand{\pthreads}{\texttt{pthreads}}
\newcommand{\specialcell}[2][c]{%
  \begin{tabular}[#1]{@{}c@{}}#2\end{tabular}}

\begin{document}

\title{NUMAlloc: Fine-Grained NUMA Memory Allocator}

\markboth{IEEE Transactions Transactions on Parallel and Distributed Systems}{NUMAlloc: Fine-Grained NUMA Memory Allocator}

\author{
\IEEEauthorblockN{
Tongping Liu, Xin Zhao, Wei Wang,
Bo Wu, Sandip Kundu
}

\IEEEcompsocitemizethanks{
\IEEEcompsocthanksitem Tongping Liu, Xin Zhao, and Sandip Kundu is with the Department of Electrical and Computer Engineering, University of Massachusetts Amherst, Amherst, MA 01003.
\IEEEcompsocthanksitem Wei Wang is with the Department of Computer Science, University of Texas at San Antonio, San Antonio, TX 78249.
\IEEEcompsocthanksitem Bo Wu is with the Department of Computer Science, Colorado School of Mines, CO 80401.}
}

%\IEEEdisplaynontitleabstractindextext
\IEEEtitleabstractindextext{%
\begin{abstract}


The NUMA architecture was proposed to accommodate the hardware trend with the increasing number of CPU cores. However, existing memory allocators have multiple issues for support the NUMA architecture: most of them were not designed for the NUMA architecture; NUMA-aware allocators still miss some performance opportunities promised by the hardware. 

This paper proposes a novel memory allocator--\NM{}---that is designed for the NUMA architecture. Different from existing allocators, \NM{} integrates the task assignment into the allocator, where each thread is bound to a specific core in an interleaved way. Based on the location of a task, \NM{} designs a node-aware memory allocation. \NM{} also utilizes huge page support to improve the performance, and designs its interleaved and block-wise memory allocation to accommodate shared objects. 
%\NM{} supports huge pages that are already available inside the OS and the hardware, and introduces an optional interleaved heap. 
Based on its extensive evaluation, \NM{} achieves up to $5\times$ performance speedup comparing to the default Linu	x allocator.  

	
\end{abstract}
\begin{IEEEkeywords}
Non-Uniform Memory Access (NUMA); NUMA Memory Allocator; Fine-Grained Memory Management;
\end{IEEEkeywords}
}

\maketitle

\date{}
%\IEEEpeerreviewmaketitle



%\todo{We need to add the memory difference with and without the transparent memory.} 
\begin{comment}
\todo{
1. verify the impact with and without origin-based deallocation (need to change few lines of code).

2. TCMalloc-version. TBB version. 

3. support the change of configurations via environment variables (binding, interleave heap). BTW, we don't need to check the size of objects any more. 

4. Memory overhead (mostly due to huge page, origin-based allocation) 
deallocations send back to the os when above some threadhold. 


Maybe we should target Micro-2022 instead. Then we should add the hardware organization figure. 

5. Let's claim our support of transparent page support as a novelty as well. 

6. check where the performance improvement is coming from, such as transparent pages, region-aware allocations, and interleaved heap.

7. If we don't ensure locality of deallocations, what is the performance impact. 
}

\end{comment}

\section{Introduction}
\label{sec:intro}

%The Non-Uniform Memory Access (NUMA) architecture is a scalable hardware design. Compared to Uniform Memory Access (UMA) architecture, the NUMA architecture avoids the bottleneck of using one memory controller, where each processor (or node interchangeably) can access its own memory controller concurrently in theory. However, it is extremely challenging to achieve the expected scalability for multithreaded applications. One notorious issue is caused by remote accesses that a task accesses the memory located in a remote node, which hurts the application performance since the latency of remote accesses is much higher than that of local accesses~\cite{Blagodurov:2011:CNC:2002181.2002182}. In addition to that, node imbalance may actually introduce the congestion of memory controllers or interconnection ~\cite{Blagodurov:2011:CNC:2002181.2002182}. Therefore, it is critical to reducing remote accesses or node imbalance for multithreaded applications. Although programmers could employ the assistance of different profiling tools to fix NUMA performance issues within applications~\cite{Intel:VTune, Memphis, Lachaize:2012:MMP:2342821.2342826, XuNuma, NumaMMA, 7847070, NumaPerf}, they cannot fix NUMA performance issues caused by a memory allocator.
%For instance, the latency of remote accesses is typically double to that of local accesses. 
Non-Uniform Memory Access (NUMA) is the de-facto design for all modern hardware in order to address the scalability issues of increasing hardware cores. In NUMA architecture, each processor (or node interchangeably) has its own memory, allowing multiple nodes to access the memory concurrently. However, it is extremely challenging to achieve the expected scalability for multithreaded applications. One notorious issue is caused by remote accesses that a task accesses the memory located in a remote node, which hurts the application performance since the latency of remote accesses is much higher than that of local accesses~\cite{Blagodurov:2011:CNC:2002181.2002182}. 
%In addition to that, node imbalance may actually introduce the congestion of memory controllers or interconnection ~\cite{Blagodurov:2011:CNC:2002181.2002182}. Therefore, it is critical to reducing remote accesses or node imbalance for multithreaded applications. 
Although programmers may employ profiling tools to identify NUMA issues of applications~\cite{Intel:VTune, Memphis, Lachaize:2012:MMP:2342821.2342826, XuNuma, NumaMMA, 7847070, NumaPerf}, they cannot fix the performance issues caused by a memory allocator.

% Based on this guide https://libraryguides.vu.edu.au/ieeereferencing/gettingstarted#s-lg-box-wrapper-9930413, it's acceptable
% \todo{The IEEE seems to have multiple references separately but I am not used to it. Double check it.}

General-purpose memory allocators, such as dlmalloc~\cite{dlmalloc},  Hoard~\cite{Hoard}, TCMalloc~\cite{tcmalloc}, jemalloc~\cite{jemalloc}, SuperMalloc~\cite{supermalloc}, and  Scalloc~\cite{Scalloc}, were designed for symmetric multiprocessing machines. As a result, they cannot achieve good performance for NUMA architecture, based on existing work~\cite{tcmallocnew, yang2019jarena} and our evaluation. There exist some NUMA-aware allocators~\cite{tcmallocnew, tcmalloc2, kim2013node, yang2019jarena, mimalloc}. In particular, Kaminski built the first NUMA-aware memory allocator on top of TCMalloc in 2008~\cite{tcmallocnew}, called \TN{} in the remainder of this paper. \TN{} adds a freelist  and a page-span for each NUMA node so that threads can allocate objects from its per-node list/page-span. To reduce remote access, \TN{} allocates the physical memory of each page-span from the same node as the current thread, where mimalloc~\cite{mimalloc} and recent NUMA support of TCMalloc follows the similar idea.  However, \textit{none of these allocators achieve the full locality of memory allocations}, as they did not handle remote access caused by memory deallocations: a freed object is typically placed into the deallocating thread's local buffer, which will violate the locality if this object is originally allocated from a remote node  (e.g., in the producer-consumer model). More importantly, they do not consider the performance impact caused by potential thread migration, and  cannot take the full advantage of huge pages that are prevalent in modern hardware. Instead, \NM{} overcomes these issues as follows. 

\text{First}, it proposes the first \textbf{binding-based memory management} to eliminate unnecessary remote accesses caused by thread migrations. A thread's migration will turn all its previously-local accesses into remote ones, and therefore can significantly degrade the performance. Although the ``thread binding'' has been frequently employed to improve the performance of applications on NUMA architecture~\cite{li2013numa, XuNuma, Lepers:2015:TMP:2813767.2813788}, it has never been the \textit{first-class citizen} for the memory allocator design. That is, existing work  binds threads of applications externally,  but the memory allocator is not aware of the binding. In contrast, the proposed work still allow users to customize the binding via the configuration file or environment variable, but \textit{will exploit the binding to improve its memory management} as follows: (1) all metadata can be allocated in the local node, based on the binding; (2) It can locate the location of threads without invoking expensive system calls, enabling full locality of memory management as discussed below. Note that \NM{}'s thread binding does not exclude the OS scheduling, as it only binds threads to the NUMA node, but not a specific core. Section~\ref{sec:limit} further discusses its design trade-off.  


%The thread binding allows \NM{} to check each node's origin quickly. Further,  the virtual address space of the heap is divided into multiple blocks, where each block is bound to a physical node, allowing to track each object's origin by the address's range. \NM{} also ensures the origin-based allocation that always allocate an object from the same node (physically) as the request thread, combining the above-mentioned mechanisms altogether to ensure the locality of allocations. 

Second, \NM{} proposes \textbf{threads-shared incremental allocation} to take advantage of ``Transparent Huge Pages'' (THP) of modern OS/hardware~\cite{hugepage}. Huge pages are expected to significantly reduce Translation Lookaside Buffer (TLB) misses, as each page table entry could cover a larger range of virtual addresses (e.g., 2MB instead of 4KB). However, most of the existing allocators~\cite{dlmalloc, Hoard, tcmalloc, mimalloc} could not take advantage of huge pages, where Scalloc even suggests that ``make sure that transparent huge pages are disabled'' due to its large memory overhead~\cite{scallochugepage}. TEMERAIRE~\cite{TEMERAIRE} and LLAMA~\cite{LLAMA} support huge pages with complicated mechanisms. Instead, the proposed work designs a simple mechanism for managing huge pages: it maps huge regions (with the size larger than a huge page) of memory initially so that these regions will be backed by huge pages physically inside the OS; Further, it utilizes the same region to serve allocations with different size classes of different threads so that each thread only allocates a small number of objects from a huge region, called as ``incremental allocation''. Incremental allocations actually reduces the memory wastes, comparing to Scalloc that always allocates a big chunk of memory to each requested thread. 
Overall, \NM{} combines the best of both worlds that it takes the performance advantage of huge pages but does not compromise its memory consumption. 
%a better balance between performance and memory consumption.

\todo{Both TEMERAIRE~\cite{TEMERAIRE} and LLAMA~\cite{LLAMA} support huge pages, but with complicated mechanisms. Instead, the proposed work designs a simple mechanism for managing huge pages: it maps a large region of memory for each node initially, and these regions will be backed by huge pages physically inside the OS. Further, it utilizes the same region to serve allocations with different size classes from different threads that running on the same node. In this way, each thread only gets a small number of objects from a huge region, called as ``incremental allocation''.}

%However, on the one side, most existing allocators, such as dlmalloc~\cite{dlmalloc}, Hoard~\cite{Hoard}, TCMalloc~\cite{tcmalloc}, jemalloc~\cite{jemalloc}, do not take advantage of THP, as they typically map small chunks of memory (e.g., multiple kilobytes) each time that will be satisfied from small pages. On the other side, some allocators, such as Scalloc~\cite{Scalloc} and mimalloc~\cite{mimalloc}, consumes up to $7\times$ (as shown in Table~\ref{tab:memory_consumption}) more memory without a THP-friendly design. 
\NM{} also has other novelty designs. (1) It proposes the \textbf{origin-aware memory management} to ensure the full memory locality built on top of its binding-based mechanism: \NM{} guarantees that a freed object will be returned to the deallocating thread \textit{only if} the object is originated from the same node that the current thread is running on; otherwise, it will be returned back to its original node. (2) It designs \textbf{an interleaved heap} for allocating heap objects of the main thread from all nodes evenly, helping reduce the congestion caused by concurrent accesses from multiple children threads. This design is inspired by existing profilers' finding that shared objects allocated from the initial/main thread are the most common source of performance degradation in the NUMA architecture~\cite{XuNuma, MemProf}. The interleaved heap will be a beneficial option for some applications. (3) It designs an efficient mechanism that could move objects between different freelists, without traversing all objects in the freelist as TCMalloc~\cite{tcmalloc}, as discussed in Section~\ref{sec:movement}. 

%. Memory locality is defined as whether an object is allocated from the local physical node of the requesting thread. Most existing NUMA-aware allocators only ensure the locality of every object's first memory allocation~\cite{tcmallocnew, kim2013node, yang2019jarena}.  Instead, \NM{} additionally ensures the locality of all freed objects, eliminating the confusion caused by memory reuse (a very common behavior). \NM{} guarantees that a freed object will not be placed into a freelist belonging to a remote node. In particular, \NM{}'s thread binding allows it to determine the thread's physical node, and \NM{}'s origin-computable design can quickly identify each object's origin by the address's range. Then a freed object will be placed into the deallocating thread's local buffer (heap) \textit{only if} it is originated from the same node that the current thread is running on; otherwise, it will be returned back to its original node. 


%\NM{} utilizes the node-interleaved thread binding (binding threads to NUMA nodes interleavedly) by default, but it also supports node-saturate thread binding that will assign the same number of threads as the number of cores to a node before assigning threads to the next node. 
%In the future, we plan to support user-defined thread binding via the environment variable or a configuration file.
%Based on our experiments on multiple allocators, the node-interleaved thread binding performs the best performance when all nodes are used to run one application.
%\textbf{Experimental methodology and artifact availability.} We have performed an extensive evaluation on synthetic and real applications, with 24 applications in total. Compared \NM{} with popular allocators, such as the default Linux allocator, TCMalloc~\cite{tcmalloc}, jemalloc~\cite{jemalloc}, Intel TBB~\cite{tbb}, Scalloc~\cite{Scalloc}, and mimalloc~\cite{mimalloc},  \NM{} is running 17\% faster than the second-best allocator (mimalloc), achieving around 19\% speedup comparing to the default Linux allocator. For the best case, \NM{} runs up to $6.4\times$ faster than the default allocator. \NM{} is much more scalable than all other allocators based on our evaluation. \NM{} is ready for practical employment, due to its high performance and good scalability. The code will be opened source using GNU GPL V2 license at \url{https://github.com/XXX} (opened upon acceptance). 

%\textbf{Limitations of the proposed approach.} The proposed work may need some user customization via configuration flags, such as interleaved heap or a specific format of thread binding, based on memory patterns of user applications. Some of these mechanisms, e.g. interleaved heap, are not universally beneficial to the performance of all applications. However, they do improve the performance of some applications. Therefore, we decide to keep such options, instead of discarding them. 

Overall, this paper makes the following contributions:

\begin{itemize}

\item It proposes a \textbf{binding-based memory management} to eliminate unnecessary remote accesses caused by thread migrations.

%\item It proposes an \textbf{origin-aware memory management} to ensure full locality of memory allocations. 

\item It proposes a \textbf{threads-shared incremental allocation} to achieve a better balance between the performance and memory consumption of huge pages. 

%\item It further proposes some other mechanism

\item It presents the design and implementation of \NM{}, which has a better performance and scalability than even widely-used commercial allocators. Overall, applications with \NM{} are running about 17\% faster than those with TCMalloc, but without using more memory.
%, such as TCMalloc, jemalloc, and Intel TBB, based on our extensive evaluation.  

\end{itemize}

%The remainder of this paper is organized as follows. Section~\ref{sec:background} introduces the OS support for the NUMA architecture and common designs of memory allocators. Section~\ref{sec:implement} focuses on the design and implementation of \NM{}. After that, Section~\ref{sec:evaluation} describes its experimental evaluation, and Section~\ref{sec:limit} discusses the limitation of \NM{}. In the end, Section~\ref{sec:related} discusses some relevant work, and then Section~\ref{sec:conclusion} concludes. 
\section{Background}
\label{sec:overview}

This section introduces some background that is necessary for \NM{}. It starts with the description of the NUMA architecture and some potential performance issues. Then it further discusses existing OS support for the NUMA architecture.  

\subsection{NUMA Architecture}

\label{sec:numa}

Traditional computers are using the Uniform Memory Access (UMA) model that all CPU cores are sharing a single memory controller, where any core can access the memory with the same latency (uniformly). However, the UMA architecture cannot accommodate the hardware trend with the increasing number of cores, since all of them may compete for the same memory controller. Therefore, the performance bottleneck is the memory controller in many-core machines, since a task cannot proceed without getting its necessary data from the memory. 

The Non-Uniform Memory Access (NUMA) architecture is proposed to solve the scalability issue, due to its decentralized nature. Instead of making all cores waiting for the same memory controller, the NUMA architecture typically is installed with multiple memory controllers, where a group of CPU cores has its memory controller (called as a node). Due to multiple memory controllers, the contention for the memory controller could be largely reduced and therefore the scalability could be improved correspondingly. However, the NUMA architecture also has multiple sources of performance degradations~\cite{Blagodurov:2011:CNC:2002181.2002182}, including \textit{Cache Contention}, \textit{Node Imbalance}, \textit{Interconnect Congestion}, and \textit{Remote Accesses}. 

\paragraph{Cache Contention:} the NUMA architecture is prone to cache contention that multiple tasks may compete for the shared cache. Cache contention will introduce more serious performance issue if the data has to be loaded from a remote node. 
 
\paragraph{Node Imbalance:} When some memory controllers have much more memory accesses than others, it may cause the node imbalance issue. Therefore, some tasks may wait more time for memory accesses, thwarting the whole progress of a multithreaded application.  

\paragraph{Interconnect Congestion:} Interconnect congestion occurs if some tasks are placed in remote nodes that may use the inter-node interconnection to access their memory. 

\paragraph{Remote Accesses:} In NUMA architecture, local nodes can be accessed with less latency than remote accesses. Therefore, it is important to reduce remote accesses to improve the performance.\\


 Based on the study~\cite{Blagodurov:2011:CNC:2002181.2002182}, node imbalance and interconnect congestion may have a larger performance impact than cache contention and remote accesses. These performance issues cannot be solved by the hardware automatically. Software support is required to control the placement of tasks, physical pages, and objects to achieve the optimal performance for multithreaded applications.  

\subsection{NUMA Support Inside OS}

Currently, the Operating System already provides some support for the NUMA architecture, especially on task scheduling, or physical memory allocation. 

For the task scheduling support, the OS provides  system calls that allow users to place a task to a specific node. One example of such system calls is \texttt{pthread\_attr\_setaffinity\_np} that   sets the CPU affinity mask attribute  for a thread. Therefore, users may employ these system calls to assign tasks on a specific memory node or even a specific core. However, programmers should specify the task assignment explicitly. 

For memory allocation, the OS provides multiple methods to support NUMA related memory management. First, the OS, such as Linux, supports the first-touch or interleaved policy~\cite{lameter2013numa, diener2015locality}. By default, the OS will allocate a physical page from the same node as a task that first accesses the corresponding virtual page, also called  first-touch policy. First-touch policy maximizes local accesses over remote accesses, but it cannot eliminate remote accesses for shared objects~\cite{yang2019jarena}. The interleaved policy helps to achieve a balanced workload  on interconnection and memory controller, avoiding the load imbalance issue described above. However, users may require to invoke a specific system call explicitly to change the allocation policy to be the interleaved policy.  Second, the OS also provides some system calls that allow users to specify the physical memory node explicitly, via system calls like \texttt{mbind}. On top of these system calls, \texttt{libnuma} provides stable APIs for controlling the scheduling and memory allocation policies, and \texttt{numactl} allows a process to control the scheduling or memory placement policy inside the user space. 

Overall, existing OS or runtime systems already provide some interfaces that allow users to control the scheduling and memory policy for NUMA architecture inside the user space~\cite{yang2019jarena}. However, they all require programmers to specify the policy explicitly, which cannot achieve the performance speedup automatically. \NM{} relies on these existing system calls to support NUMA-aware memory allocations explicitly, as further described in Section~\ref{sec:implement}. 

%First, it won't work for producer-consumer model, where the memory will be utilized by threads that are not in the same node. This is actually very common, especially for the main thread. Second, it does not work when cores on different processors have to be used, if the application utilizes more physical memory of one node, or if threads are migrated later. Third, it requires the user-level memory manager support. Otherwise, an deallocated object can be allocated to threads running on the other node, causing many remote accesses. 



%Besides this, Operating System typically provides another memory policy where the memory will be allocated in an interleaved way. Also, OS provides different system calls that allow users to specific the memory policy, move memory, and schedule the threads. However, it is not designed for a specific applications, but providing interface that allows programmers to customize for their specific application. 


%Overall, most of such support cannot benefit the performance of programs automatically, which will require programmer effort to tell OS.   

 
\section{Design and Implementation}
\label{sec:implement}

\NM{} is designed as a drop-in replacement for the default memory allocator. It intercepts all memory allocation/deallocation APIs via the preloading mechanism. Therefore, there is no need to change the source code of applications to employ \NM{}, and there is no need to use the custom OS or hardware. 

Different from existing work, \NA{} aims to reduce remote accesses, and balance the workload among different hardware nodes. It also utilizes huge page support to improve the performance, and designs its interleaved and block-wise memory allocation to accommodate shared objects. Multiple components that differentiate it from existing allocators are discussed in the remainder of this section. 
  
%\NA{} also borrows many known mechanisms of existing allocators. First, it utilizes the size class to manage objects. Instead of allocating the exact size, \NA{} will round the size of an allocation to its closest size class. Similar to TcMalloc~\cite{tcmalloc}, \NA{} also utilizes fine-grained size classes for small objects, such as 16 bytes apart for objects less than 128 bytes, and 32 bytes apart for objects between 128 bytes and 256 bytes, then power-of-2 sizes afterwards. Second, it utilizes the ``\textbf{Bi}g-\textbf{B}ag-\textbf{o}f-\textbf{P}ages'' mechanism that all objects in the same bag will have the same size class, and separates the metadata from actual objects. \NA{} only tracks the size information of each page, which helps reduce its memory overhead for the metadata. Third, \NA{} utilizes freelist to manage freed objects. Every freed object will be added into a corresponding freelist, and objects in the freelist will be allocated first in order to reduce possible cache misses. Further, \NA{} utilizes the first word of freed objects to link different objects, which is similar to Linux and TcMalloc~\cite{tcmalloc}. This mechanism helps to reduce the memory overhead, but is prone to memory vulnerabilities, such as buffer overflows and double-frees~\cite{DieHarder, Guarder}.     

\begin{figure*}[h]
\begin{center}
\includegraphics[width=0.8\textwidth]{figure/heaplayout}
%\includegraphics{figure/overview2}
\end{center}
\vspace{-0.1in}
\caption{Overview of \NA{}'s Heap Layout.
\label{fig:overview}}
\vspace{-0.1in}
\end{figure*}

\subsection{Topology Aware Task Assignment} 
\label{sec:taskassign}

Existing allocators typically do not schedule tasks explicitly, but relying on the default OS scheduler. The OS scheduler performs very well for the UMA architecture, since the latency of accessing the memory controller is the same for every core. However, the NUMA architecture imposes additional challenge~\cite{Majo:2015:LPC:2688500.2688509}. If a task is moved to a new node, it has to access all of its memory remotely, resulting in a higher memory latency. The scheduling may lead to significant performance difference for memory-intensive applications. Therefore, typically tasks are bound to a specific core/processor for the NUMA architecture~\cite{terboven2012assessing, terboven2012task, Majo:2015:LPC:2688500.2688509}.  

%Similarly, \NA{} embeds the task assignment into its memory management. 
Due to the importance of task assignment to  memory locality, \textit{\NA{} embeds a topology-aware task assignment, which makes it different from existing allocators}. Every thread is bound to a specific node, where its memory allocations will be based on the location of its task, as described in Section~\ref{sec:nodeaware-memory}. To balance the workload of each node, \NA{} utilizes a round-robin manner to assign the tasks to different nodes, which is similar to TBB-NUMA~\cite{Majo:2015:LPC:2688500.2688509}. Basically, a newly-created thread will be assigned to a node that is different from its preceding and subsequent sibling. This ensures that each processor/node will have a similar number of threads, and therefore a similar workload. An alternative approach is to assign the same number of threads as cores to one node, and then assign subsequent threads to the next node. However, this approach has two issues. First, it may cause significant performance issue for applications with the pipeline model due to remote accesses, if threads of different stages are assigned to different memory nodes. Second, it may not fully utilize all memory controllers, if the number of threads is less than the number of cores in total.    

In the implementation, \NA{} recognizes the hardware topology via the \texttt{numa\_node\_to\_cpus} API, which tells the relationship between each CPU core and each memory node. It intercepts all thread creations in order to bind a newly-created thread to a specific node. \NA{} employs \texttt{pthread\_attr\_setaffinity\_np} to set the attributes of a thread, and passes the attribute to its thread creation function. Therefore, every thread is scheduled to the specified node upon the creation time. Note that a thread is pinned to a node in \NM{}, instead of a core, which still allows the OS scheduler to perform the load balance when necessarily. 

\subsection{Node-Aware Memory Management} 
\label{sec:nodeaware-memory}

\NA{} designs a node-aware memory management based on its task assignment. Since a memory allocator only deals with virtual memory, but not physical memory, \textit{\NA{} binds a range of virtual memory to a particular node via the \texttt{mbind} system call}. As shown in Fig.\ref{fig:overview}, each node is mapped to four terabytes (TB) initially, and the interleaved heap is placed in a separate place (Section~\ref{sec:mainthread}).  Therefore, \NA{} is able to compute the physical node information based on a virtual address. More specifically, it could divide the offset (from the heap beginning) by the size of each node to compute its node index. This design takes the advantage of the huge address space of 64-bits machine to achieve the quick lookup.

Each per-node heap is further divided into two parts,  one for small objects (managed with the \texttt{bpSmall} bump pointer) and one for big objects (with the \texttt{bpBig} pointer). An object with the size larger than 512 KB will be treated as a big object, and others are small objects. The space for big objects are separated from that for small objects so that we could utilize huge pages for big objects, as discussed in Section~\ref{sec:hugepage}. Another difference between small objects and big objects is that small objects are managed by size classes, 
while big objects of each node are tracked in a global free list.  For small objects, \NM{} utilizes the ``\textbf{Bi}g-\textbf{B}ag-\textbf{o}f-\textbf{P}ages'' (BiBOP) mechanism that all objects in the same bag will have the same size class, and separates the metadata from actual objects. For each size class, \NM{} maintains a freelist for each node and for each thread. Every 
size class has 16-byte difference when the size is less than 128 bytes, and then 32 bytes apart for objects less than 256 bytes. When an object is larger than 256 bytes, its size class will be always power of two. In order to track the size information of objects, \NM{} utilizes one megabytes (MB) as a basic unit, where the information is tracked in a PerMBInfo data structure (as shown in Fig.~\ref{fig:overview}. That is, a big object will be always aligned to megabytes, and small objects in the same MB (called as a bag) will has the same size. For a big object that is larger than 1MB, all corresponding bytes in the PerMBInfo will be set to the specific size upon the allocation. \NM{} embeds the availability information of each chunk to the \texttt{PerMBInfo} structure, using the lowest significant bit of the size. \NM{} is able to coalesce multiple continuous big objects into a bigger object upon deallocations.  


Except those ones stated in Section~\ref{sec:mainthread}, \NA{} ensures node-aware memory allocations that an allocation is guaranteed to be allocated from the local node physically.
\textit{For each deallocation, \NM{} ensures that an object is always placed into a freelist of its original node}. More specifically, for small objects, it will be  placed into the current thread's freelist \textbf{only if} it is originated from the same node that the current thread belongs to. Otherwise, it will be placed to the per-node freelist of its original owner. A big object is always placed into the freelist of its owner. \textit{\NM{} also handles the allocation carefully to ensure the local allocation}. For small objects, an allocation will be satisfied from the freelist of the current thread, where all freed objets is originated from the local node, if there exists some freed objects. Otherwise, the thread will migrate some freed object from the per-node freelist. If its per-node freelist is empty, then it will get a batch of never-used objects from the bag of a size class. Allocations of big objects is always satisfied from per-node freelist first, before getting a new object from the per-node heap.   


%Objects in the freelist will be allocated using the last-in-first-out order, which helps reduce possible cache misses since last-freed objects are more likely to be hot. A big object will be allocated from the per-node freelist, while a small object will be allocated from the thread's per-thread list if possible and then the per-node freelist. Due to this reason, \NM{} moves small objects between per-thread lists and per-node lists: if freed objects in a per-thread list is over a pre-defined threshold, a percentage of freed objects will be moved to the thread's per-node list; Whenever the per-thread list is empty, some objects will be moved from the corresponding per-node freelist. \NM{} also designed a mechanism to move objects between different freelists, as further described in Section~\ref{sec:moveobjects}. 
%Both movement will take an adaptive mechanism. For instance, if a per-thread list keeps migrating objects from its per-node list successfully, then the percentage of moving to per-node list will be reduced. Similarly, if a per-node list do not have objects to move to the per-thread list, \NM{} will move less frequently.  

%An object with the size larger than 512KB will be treated as a big object, and others are small objects. As described above, \NM{} maintains multiple size classes for small objects. Upon each deallocation, \NA{} determines its physical node (based on its virtual address) and the size information in order to place it correspondingly. For small objects, if an object is allocated from a different node, it will be placed into the common freelist of that node. That is, an object is always guaranteed to be allocated locally. Otherwise, it will be placed into the freelist of the deallocation thread. Utilizing a per-thread list avoids the synchronization overhead, since only one thread is allowed to access its per-thread list. 

%In its implementation, \NA{} reduces the overhead of frequent system calls, and borrows the information-computable design from existing allocators in order to quickly locate the physical node information~\cite{FreeGuard, Guarder}. The basic idea of its memory layout is illustrated in Fig.~\ref{fig:overview}. 
%Basically, \NA{} takes the advantage of the huge address space of 64-bits machine to achieve the quick lookup. Instead of frequently allocating the memory from the OS, \NA{} obtains a big chunk of memory (few terabytes) from the underlying OS initially, and then divides it to multiple chunks with the same size (1MB), where each chunk will be bound to a specific memory node as illustrated in Fig.~\ref{fig:overview}. Therefore, \NM{} only requires to record the size information for each megabytes, shown as \texttt{PerMBInfo} in Fig.~\ref{fig:overview}. The physical node information could be computed directly from the virtual address, by checking the distance from the starting address of the heap. Then \NM{} is able to quickly locate the size information by checking the  Comparing the method of using a hash table to track the node information~\cite{tcmallocnew}, this mechanism is much faster, since it avoids the hashing and lookup overhead.  

%The memory of each node is further divided into two parts, one for small objects (managed with the \texttt{bpSmall} bump pointer) and one for big objects (managed using the \texttt{bpBig} pointer). There are two reasons to separate them. First, we plan to utilize huge page support for big objects, as further described in Section~\ref{sec:hugepage}.  Second, placing big objects together helps to coalesce two continuous objects into one bigger object. For small objects, \NM{} employs the ``\textbf{Bi}g-\textbf{B}ag-\textbf{O}f-\textbf{P}ages'' (BiBOP) style that each bag will have multiple continuous pages, and each bag will have objects with the same size class. \NM{} currently sets its bag size of small objects to be one megabytes. An object with the size larger than 512 kilobytes is considered to be a big object, which will be always aligned to megabytes. Therefore,  Overall, there is a freelist for all big objects for each node. But for small objects, there are  

%\NM{} embeds the availability information of each chunk to the \texttt{PerMBInfo} structure, using the lowest significant bit of the size. Each node has  a per-node heap, which includes a per-node freelist to track big objects, and multiple freelists to track small objects with different size classes. As described above, each thread also has its freelists for different size classes. Small objects can be moved between a per-thread freelist and a per-node list as described above. 
 

%When a big object is freed, \NM{} checks the availability of its previous and next neighbor, and performs the coalesce if its neighbor is also available. \NM{} re-utilize a freed big object for a bag of small objects, since the bag size of small objects and big objects are both aligned to one megabytes. This is different from most existing allocators, which typically returns big objects to the OS upon deallocation. Comparing to them, \NM{} reduces cache misses by re-utilizing big objects.  

\textit{Cache Warmup Mechanism:} For small objects, \NM{} also borrows the cache warmup mechanism of TcMalloc~\cite{tcmalloc}. TcMalloc utilizes the \texttt{mmap} system call to obtain multiple pages (depending on the class size) from the OS each time, when it is running out of the memory for one size class. For such a memory block, TcMalloc adds all objects of this block into its central freelist at one time. Since TcMalloc utilizes the first word of each object as the pointer for the freelist, this mechanism warms up the cache by referencing the first word of each object during the adding operation. According to our observation, this warmup mechanism improves the performance of one application (\texttt{raytrace}) by 10\%. Based on our understanding, the performance improvement is caused by data prefetches, since adding objects to the freelist has a simple and predictable pattern. \NM{} employs a similar mechanism for small objects with the size less than 256 bytes, and adds all objects inside a page to the per-thread freelist. 
%A bump pointer is used to track the position of never-used objects, whose updates do not require to be exactly one page each time. By comparison, TcMalloc may waste some memory in the end of each block, if the block size is not equal to multiple times of the class size.   

\subsection{Interleaved Heap} 
\label{sec:mainthread}

\NA{} proposed a new interleaved heap for shared objects. Based on our observation, most NUMA performances issues identified by existing NUMA profilers are related to shared objects~\cite{XULIU, MemProf}. These shared objects are typically allocated in the main thread, but are  passed to children threads later. Allocating the physical memory for shared objects in one node may introduce load imbalance issue, and therefore causing significant number of remote accesses. Therefore, \NA{} reserves a range of memory for shared objects, called as ``Interleaved Heap'' in Fig.~\ref{fig:overview}. \NA{} utilizes the \texttt{mbind} system call to specify that the physical pages of this range will be allocated from all nodes interleavedly. This design helps balance the volume of memory accesses of all memory controllers, reducing interconnect congestion and load imbalance. 

\begin{comment}
\begin{wrapfigure}{r}{0.6\textwidth}
\centering
\includegraphics[width=3in]{figure/blockwise}
\vspace{-0.1in}
\caption{Block-wise Memory Allocation\label{fig:blockwise}}
\vspace{-0.1in}
\end{wrapfigure}
\end{comment}

%\NA{} allocates the memory in a block-wise way for shared objects, when the size of an allocation is larger than the twice of the multiplication of nodes and the page size. The basic idea of block-wise allocation is illustrated in Fig.~\ref{fig:blockwise}. The key reason to use the block-wise allocation is that children threads may access the memory continuously~\cite{XULIU}. In addition to that, the block-wise allocation will still maintain the balance between different memory nodes, which won't cause unnecessary performance issue. 

%Otherwise, it will introduce performance issue by allocating private objects in a different node, causing remote accesses unnecessarily.
 
During the implementation, \NA{} only allocates shared objects from the main thread in the interleaved heap. Although it may benefit the performance, if shared objects allocated in children threads are also using the interleaved heap. However, the overhead of tracking shared objects is typically larger than the performance benefit based on our evaluation (skipped in the paper). 
%, especially for applications with the pipeline model, the overhead of tracking shared objects is typically larger than the performance benefit based on our evaluation (skipped in the paper). 

It is very important to identify shared objects correctly and efficiently. \NA{} utilizes the allocation/deallocation site to identify shared objects: an object is treated as a shared one initially, and is allocated from the interleaved heap. This method is aligned with the intuition, since the shared behavior of a callsite is determined by the program logic. When a newly-allocated object is deallocated before creating children threads, all objects from the corresponding callsite are considered to be private objects. 
%Therefore, \NA{} utilizes the callsite to differentiate objects, which is determined by the program logic.  

%However, it is very challenging to determine the allocation callsite \textit{uniquely} and \textit{efficiently} due to multiple reasons. First, function pointers are removed in most applications if they were compiled with an optimization, which makes it impossible to obtain return addresses (and callsite) via built-in functions. Second, applications have a wrapper for memory allocations, which requires multiple levels of call stacks to uniquely identify one callsite. Although there exist mechanisms to encode calling context explicitly~\cite{DBLP:conf/icse/SumnerZWZ10, DBLP:conf/cgo/ZengR0AJ014}, they require the recompilation of the applications. The \texttt{backtrace} function can obtain the callsite, but it is too slow to be used in production environment.

For the performance reason, \NA{} further proposes to utilize the sum of the stack position and the return address of the allocation to identify a callsite, called ``\textit{callsite key}''. 
  When memory allocations are invoked in different functions, their stack positions are likely to be different. The return address (of the application) tells the invocation placement inside the same function. However, this design cannot completely avoid mis-identification issue,  when there exists the allocation wrapper and multiple allocations inside the same function will be treated as the same callsite. But the mis-identification will not cause any correctness issue. During the implementation, we have thought about two other mechanisms. One is to obtain the callsite correct with the \texttt{backtrace} function, but it is too slow to be used in production environment. The other mechanism will require the recompilation to encode calling context explicitly~\cite{DBLP:conf/icse/SumnerZWZ10, DBLP:conf/cgo/ZengR0AJ014}.
  
%   A shared callsite can be treated as a private callsite, if an object from  these callsites invokes the deallocation before creating children threads. For the performance reason, \NA{} obtains the return address quickly via a constant offset, where the offset is uniquely determined after the compilation of \NA{}. 

\NA{} utilizes a hash table to track the status of every callsite. Upon every allocation of the main thread, \NA{} checks the status of the allocation callsite, with the callsite key as described above. If the callsite is identified as a shared callsite, the current allocation will be satisfied from the interleaved heap. Otherwise, it will be allocated from the per-node heap. Upon deallocations, \NM{} will  mark an allocation callsite as private, if an object from this callsite has been deallocated before creating other threads. 

%When an allocation is satisfied in the callsite and the callsite is new, the corresponding allocation will be tracked in the second hash table. The second hash table will be checked upon deallocations, where the corresponding allocation callsite will be marked as private if an object is allocated in the same epoch. 

\subsection{Explicit Huge Page Support} 
\label{sec:hugepage}
%Modern hardware typically installs with huge page support.
\NA{} employs explicit huge page support to further improve the performance. The size of a huge page is 2 megabytes (MB) in the current OS. Based on the existing study~\cite{hugepages}, huge pages can reduce Translation Look-aside Buffer (TLB) misses that may significantly affect the performance, since a huge page covers a larger range of memory than a normal page (4 KB). Reducing TLB misses helps reduce the interferences on the cache utilization caused by TLB misses, and therefore reduces cache misses. Huge pages could also reduces the contention in the OS memory manager, since it requires much less time on page table management. We observed around 10\% performance improvement for some applications, when we are using huge page support. 

However, existing study showed that the transparent huge page support of the OS is not good for the performance~\cite{Gaud:2014:LPM:2643634.2643659, DBLP:conf/asplos/PanwarBG19}. In fact, it may have some harmful impact on the performance of NUMA systems. First, it can cause the \textit{hot page effect} when multiple frequently-accessed objects  were  mapped  to  the  same  physical  page, causing the overloading of the corresponding memory node. Second, huge pages are more prone to \textit{page-level false sharing}, when multiple threads are accessing different data inside the same page. Besides that, the huge page may increase the memory footprint~\cite{DBLP:conf/asplos/MaasAIJMR20}, if a partial range of a huge page is not accessed. 

\NA{} utilizes huge pages explicitly to avoid these issues. Ideally, if huge pages are only utilized for private objects, then there are no hot page effect and page-level false sharing. Also, if huge pages are only utilized for big objects or all small objects in a huge page will be allocated, then there is no need to worry about the memory consumption. \NA{} takes these consideration into account. In addition to large objects that has the size larger than the size of a huge page, huge pages are only utilized for small objects that are predicted to be used a lot.  \NA{} employs the history of memory allocation to predict this. 
%Each per-node heap is further divided into two parts as illustrated in Fig.~\ref{fig:overview}: small objects will be allocated from the first half and will be allocated using small pages, while big objects will be allocated from the second half with huge pages (2MB). When a big object (with the huge page) is utilized for small objects, only frequently-allocated small objects can utilize such an object. We believe that our design balances the performance and memory consumption.   

\subsection{Efficient Objects Migration} 

\label{sec:moveobjects}

\NM{} maintains per-thread heap and per-node heap in order to reduce the synchronization overhead, which indicates the necessity of moving freed objects between different freelists. 
%All threads bounded to the same node will share the per-node heap. 
When a per-thread list has too many objects, some of them should be moved to the per-node list so that other threads could re-utilize these freed objects, reducing the memory consumption. Similarly, each per-thread list may need to obtain freed objects from its per-node heap, when a thread is running out the memory. These frequent migrations require an efficient mechanism. Currently, \NM{} utilizes singly link lists to manage freed objects, imposing an additional challenge of migrating objects efficiently. 

One straightforward method is to traverse the whole list to collect a batch of objects, and then moves them at a time. But this method has multiple issues.
%Although the idea of using the batch reduces the contention, but this still has multiple issues. 
First, traversing a list will actually pollute the cache of the current thread, when a thread is moving out objects from its per-thread heap, since \NM{} utilizes the first word of each object as the pointers for the linked list. That is, the current thread will not need these objects in the future, where traversing these objects will bring these objects to the cache. 
Second, it is very slow to traverse a list, which may introduce thread contention when multiple threads are migrating freed objects from the per-node heap concurrently. We observed a 20\% slowdown for some applications due to this straightforward mechanism. \NM{} further proposes multiple mechanisms to migrate objects efficiently. 

\begin{wrapfigure}{r}{0.6\textwidth}
\centering
\includegraphics[width=3in]{figure/perthreadlist}
\vspace{-0.1in}
\caption{Avoiding the traverse of per-thread freelist\label{fig:perthreadlist}}
\vspace{-0.1in}
\end{wrapfigure}
In order to avoid the pollution on the per-thread cache, each per-thread freelist maintains two pointers that pointing to the least recent object (shown as the \texttt{Tail} object) and the $nth$ object separately, as shown in Figure~\ref{fig:perthreadlist}. Therefore, a thread can migrate $n$ objects (between $(n-1)th$ and the \texttt{Tail} object) easily. 
%It only requires a forward check to obtain the pointer of $(n-1)th$ object and then it could move out $n$ objects at a time. 
After the migration,  the \texttt{Tail} pointer can be set to the original $nth$ object. But this mechanism alone cannot reduce the contention when multiple threads are concurrently obtaining objects.

\begin{wrapfigure}{r}{0.6\textwidth}
\centering
\includegraphics[width=3in]{figure/listarray}
\vspace{-0.1in}
\caption{An array of freelists for per-node heap\label{fig:listarray}}
\vspace{-0.2in}
\end{wrapfigure}
In order to avoid the bottleneck of the per-node heap, \NM{} introduces a circular array of freelists as shown in Figure~\ref{fig:listarray}, where the number of entries is a variable that can be changed by the compilation flag. Each entry has two points, head and tail separately. In order to support the put and get operations to the freelist, this array has two pointers, \texttt{toGetIndex} and  \texttt{toPutIndex}. If a thread tries to obtain freed objects from the per-node heap, it will obtain all objects in the freelist pointed by the \texttt{toGetIndex}, and update the index afterwards. If the freelist pointed by the \texttt{toGetIndex} has no freed objects, there is no freed objects in the per-node heap for this size class.  The put operation will utilize the pointer \texttt{toPutIndex}. As described above, there are two scenarios for the put operation. First, a thread may put an freed object to the per-node freelist, if this object is originated from a different node. In this case, the object will be placed into the freelist pointed by the \texttt{toPutIndex}, but the index is not updated after the deallocation. Second, when freed objects in a per-thread freelist is above the predefined watermark, the thread will migrate a batch of objects to the freelist pointed by the \texttt{toPutIndex}. After this migration, the current freelist is considered to be full, and will update the index to the next entry in the circular array.

   

%When a thread migrates freed objects to the per-node heap, it will add the list of objects to the freelist pointed by \texttt{toPutIndex}, and update the index after the migration. Similarly, iAs described above, if a thread running on the other node just deallocates an object to the per-node heap, then this object will be added to the head of the freelist pointed by  \texttt{toPutIndex}, but the index is not changed after the deallocation.  
 

\subsection{Other Mechanisms}
\label{sec: others}

\paragraph{Node-Local Metadata:} \NM{} guarantees that all of the metadata is always allocated in the same node, based on its task binding as described in Section~\ref{sec:taskassign}. Such metadata includes per-node and per-thread freelists for different size classes, and freelists for big objects. Similarly, \NM{} utilizes the \texttt{mbind} system call to bind the memory to a specific node.  

\paragraph{Memory Reutilization:} Some applications may create new threads after some threads have exited. \NM{} re-utilizes the memory for these exited threads. Basically, \NM{} introduces a thread index for each thread, which is utilized to index the corresponding per-thread heap.  \NM{} intercepts thread joins and cancels so that it can assign the indexes of exited threads for newly-created threads, and re-utilize their heaps correspondingly.  


\paragraph{Transparent Huge Page Support:} During the development, we noticed that excessive memory consumption can be imposed when the OS enables transparent huge pages by default. In order to reduce memory consumption, \NM{} makes multiple threads share the same bag for the same size class, instead of having a separate bag for each thread. Each thread will get a number of objects from the bag, if it runs out of the memory for a size class. Currently, if a class size is less than one page, then we will at most get objects with the total size of one normal page. Otherwise, it will get 4 objects (with the size less than 64 KB) or 2 objects afterwards.  

%we propose the combination of per-node heap and per-thread cache. In order to reduce the contention, \NM{} will obtain multiple objects at a time from the per-node heap. 

 
%https://queue.acm.org/detail.cfm?id=2852078


\section{Experimental Evaluation}
\label{sec:evaluation}

This section aims to answer the following research questions: 

\begin{itemize}
\item \textbf{Performance:} How is \NM{}'s performance on synthetic benchmarks and real applications, comparing to popular allocators and NUMA-aware allocators? (Section~\ref{sec:performance}) 
\item \textbf{Memory Consumption:} What is the memory consumption of \NM{}? (Section~\ref{sec:memory})
\item \textbf{Scalability:} How is the scalability of \NM{}? (Section~\ref{sec:scale})
\item \textbf{Design Decisions:} How important design choices can actually affect the performance? (Section~\ref{sec:design})	
\end{itemize}

\subsection{Experimental Setup}
\NM{} was evaluated on two different machines, as specified in Table~\ref{table:Machine}. Typically, machine A has 2 nodes, with 40 cores in total, while machine B has 8 nodes with 128 cores. For machine B, any two nodes are less than or equal to 3 hops. For the evaluation, both machines turned off the hyperthreading. For the performance data, all data shown in this paper is the average of 10 runs, in order to avoid any bias caused by unexpected events.  

\begin{table}[!ht]
 \centering
  \footnotesize
  \setlength{\tabcolsep}{1.0em}
\begin{tabular}{c c c}
\hline
System & \textbf{Machine A} & \textbf{Machine B} \\ \hline
CPUs/Model & Xeon Gold 6138	& Xeon(R) Platinum 8153\\ \hline
CPU Frequency & 2.10GHz & 2.00GHz\\ \hline
NUMA Nodes & 2 & 8 \\ \hline
Physical Cores & 2$\times$20 & 8$\times$16 \\ \hline
Node Latency & \specialcell{local: 1.0 \\ 1 hop: 2.1} & \specialcell{local: 1.0 \\ 1 hop: 2.1 \\ 2 hops: 3.1}\\ \hline
Interconnect Bandwidth & 8GT/s & 10.4GT/s\\ \hline
Linux & Ubuntu 18.04 & Debian 10\\ \hline
Compiler & GCC-7.5.0 & GCC-8.3.0 \\ \hline
%Memory Bandwidth & 19.87 GB/s & \\ \hline
  \end{tabular}
   \caption{Machine Specifications.\label{table:Machine}}
  \vspace{-0.4in}
\end{table}


\subsection{Performance Evaluation}

\label{sec:performance}

In order to evaluate the performance, we employ both synthetic applications (in Section~\ref{sec:scale}) and real applications on two different machines, where all are multithreaded applications. The number of threads is set to the the total number of cores on two machines if possible, with 40 threads in machine A and 128 threads in machine B. For applications with multiple phases (e.g., \texttt{ferret} and \texttt{dedup}) or works only for power-of-two threads, we chose the maximum number of threads that is smaller than but is close to the total number of cores.  We compare \NM{} with multiple popular allocators, such as default Linux Allocator, TcMalloc-2.7~\cite{tcmalloc}, NUMA-Aware TcMalloc~\cite{tcmallocnew}, jemalloc-jemalloc-5.2.1~\cite{jemalloc}, Intel TBB--2020.1~\cite{tbb}, and Scalloc-1.0.0~\cite{Scalloc}. For the simplicity, NUMA aware TcMalloc is called as TcMalloc-NUMA in the remainder of this paper. The performance data is using the normalized runtime, by normalizing the runtime of each allocation to the runtime of Linux's default   allocator. That is, the lower bar indicates a better performance. In the remainder of this paper, all performance data are using the same format. 

For real applications, we evaluated on all applications from the PARSEC suite~\cite{parsec}, and seven real applications like \texttt{Apache httpd-2.4.35}, \texttt{MySQL-5.7.15}, \texttt{Memcached-1.4.25}, \texttt{SQLite-3.12.0}, \texttt{Aget}, \texttt{Pfscan}, and \texttt{Pbzip2}. 
The inputs for these applications are listed as follows. PARSEC applications are using native inputs~\cite{parsec}. For MySQL, we use \texttt{sysbench} with 40 and 128 threads separately, each issuing 100,000  requests. The \texttt{python-memcached} script is used for \texttt{Memcached}, with 3000 loops to get the sufficient runtime~\cite{memcached}. The  \texttt{ab} is used to test \texttt{Apache} server~\cite{apachetest}, by sending 1,000,000 requests. \texttt{Aget} is tested  by downloading a 30 M file, and \texttt{Pfscan} is tested by searching  a keyword in a 500M data. In terms of \texttt{Pbzip2}, we test it by compressing 10 files with 30M each. Finally, SQLite is tested through a program called \texttt{threadtest3}~\cite{sqlitetest}. 

%In the Hoard~\cite{Hoard} benchmarks, we used 100 iterations and 1,280,000 64-byte objects for threadtest and also we run larson for 10 seconds with 1,000 7-2048 bytes object to cover all size classes in almost all allocators for 10,000 iterations.For false sharing , we used 100,000 inner-loop , 100,000 iterations with 8 bytes objects. 

%The number of threads of all benchmarks were adjusted according how many cores and nodes in the target machine to make threads could be properly distributed over the nodes and cores, making the number of threads as close as the number of cores. Mostly, thread number was 40 in the Machine A and 128 in the Machine B, and I will give the specific number below if it is not this default value. 
The normalized performance runtime of different allocators on Machine A an on Machine B can be seen in Fig.~\ref{8node-parsec-perf} and Fig.~\ref{2node-parsec-perf} separately,  where all data is normalized to the runtime of the default Linux allocator. By default, \NM{} will embed with the interleaved heap support. However, two applications, \texttt{canneal} and \texttt{raytrace}, have  a much worse performance when the interleaved heap is enabled, since both of them spend a large portion of their time (over 62\% and 82\%) in the serial phase (before creating any child thread). Since the interleaved heap indicates that the allocations can be satisfied in remote NUMA nodes, this design may lead to a large number of remote accesses for the serial phase. Thus, these two figures show the best data for two applications, without the support of interleaved heap. We further discuss the pros and cons of using interleaved heap in Section~\ref{sec:interleavedheap}.  


\begin{figure}[H]
    \centering
    \begin{subfigure}{0.9\textwidth}
    \includegraphics[width=\textwidth]{figure/2-node-parsec-perf.pdf}
    \caption{Machine A (2-node)\label{2node-parsec-perf}}
    \end{subfigure}
    
	\vspace{0.1in}  
	
	\begin{subfigure}{0.9\textwidth}    \includegraphics[width=\textwidth]{figure/8-node-parsec-perf.pdf}
    \caption{Machine B (8-node)\label{8node-parsec-perf}}
    \end{subfigure}
    \caption{Normalized performance overhead for different allocators \label{sec:perf}}
 \end{figure}


Overall, \NM{} has the best performance on two machines. Comparing to the default allocator, \NM{} is 6\% faster on Machine A and 13\% faster on Machine B. TcMalloc is the second best one among all allocators, which is only 1\% faster than that of the default allocator. \NM{} is actually much faster than the other NUMA-aware allocator -- TcMalloc-NUMA~\cite{tcmallocnew}. For the best case (e.g., \texttt{fluidanimate}), \NM{} is running up to  $5.8\times$ faster than the default Linux allocator, and it is $4.7\times$ than the second best one--TcMalloc. We also notice that \NM{} achieves a much better performance on the machine with more hardware cores and more NUMA nodes, which indicates \NM{}'s scalable design. 

The default Linux allocator achieves a reasonable performance on the NUMA architecture due to its arena-based design. Based on our analysis, the Linux allocator will always return an object back to its original arena, and then allocate such objects to the thread owning this arena afterwards. This design is integrating well with Linux's first-touch allocation policy, which essentially avoids the owner shifting issue of most allocators. By default, Linux utilizes the first-touch policy to manage the physical memory~\cite{Lameter:2013:NO:2508834.2513149}, which a page is allocated in the same node as the thread that first touches it. Therefore, an object is typically allocated from the local node of its allocation thread, since it typically accesses this object after the allocation. Objects that are deallocated from a different thread will be always returned back to its original arena, and then will be re-utilized by its original allocation thread locally. In contrast, other allocators typically utilize a per-thread cache to store objects that are deallocated by the current thread, which may lead to remote accesses unnecessarily in a NUMA architecture when the deallocation thread is located in a different node from the allocation thread, causing the ``owner shifting'' issue.  

 TcMalloc-NUMA is the only available allocator that is claimed to support the NUMA architecture~\cite{tcmallocnew}. However, its performance is not good, which is even slower than that of TcMalloc. TcMalloc-NUMA is based on TcMalloc-0.97 (released in 2008), which does not have many new features of TcMalloc-2.7 (the version for our evaluation). TcMalloc-NUMA imposes the largest overhead \texttt{fluidanimate} ($2.04\times$) on the Machine B (8-node). \todo{Is it caused by thread binding?} Based on our understanding, although it achieves node-aware memory management, it does not implement following mechanisms of \NM{}, including topology-aware task assignment, interleaved heap, automatic huge page support, and efficient object migration. We examine the performance impact of these mechanisms further in Section~\ref{sec:design}.    
  


%We  can see that the average value of \NM{} is 0.97 in Machine A and 0.92 in Machine B and it is always the best among all other allocators. The reason that \NM{} got better performance in Machine B is that there are more nodes and more cores in Machine B, which means \NM{} could be very helpful to better to take use hardware resource of multi nodes and cores. but we could get amazing improvement if we shutdown interleaved heap in \NM{} and we will give the data in following sections.In the figure ~\ref{8node-parsec-perf}, we could see more exciting improvement from \NM{}, with average normalized value of 0.92 that is not only the best but also far aware better than all the rest allocators that TcMalloc and jemalloc got 0.99, TcMalloc-NUMA and TBB got roughly 1.07 and 1.01 separately. And also, we can see that the performance of \NM{} is the best for almost each single applications, especially it got 0.17 in fluidanimate and 0.66 in streamcluster which is far better than any of other allocators. As the same thing, the performance of ratrace and canneal is not good here, we will talk about it later after we shut down the interleaved heap.


%In the figure ~\ref{hoard-perf}, we show the normalized performance for Hoard benchmarks in Machine A and Machine B separately. We can see from figure ~\ref{hoard-perf} that the average value of \NM{} is also the best, which is 0.47 that means 2 times faster than default Linux Allocator, and jemalloc got 0.7 and Scalloc got 0.9. In the threadtest, the normalized value of \NM{} is 0.19 , far better than any of others, which means there are few central free list competitions, mainly contributed by properly node management and low overheads operations. For false sharing, \NM{}'s performance is also almost the best as same as Scalloc and jemalloc, which means they could handle false sharing issues very properly. In the larson, \NM{} and TcMalloc are the best, which mainly contributed by their low overheads for allocation and remote de-allocation, but due to our better node management, \NM{} could be better in the Machine B which will be mentioned later. In the figure ~\ref{hoard-perf}, we can also see that \NM{} got lowest average normalized value:0.33, significantly smaller than any of others that TBB got 0.99, Scalloc and jemalloc got roughly 1.14. And also, \NM{} and Scalloc could handle false sharing issue very well, and \NM{} could extremely well reduce central free list competition in threadtest. In larson, \NM{} is the best due to its properly multi-node management. 


\subsection{Memory Consumption}
\label{sec:memory}

We also measured the maximum memory overhead of different allocators for these applications. For non-server applications, such as \texttt{Aget}, \texttt{Pbscanf}, \texttt{PbZip2} and all PARSEC applications, we utilized the sum of the maxresident output from the time utility and the size of huge pages, since the time output does not include the usage of huge pages. In order to determine the huge page usage, a script is used to periodically collect the number of huge pages by reading from \texttt{/proc/meminfo} file, and then the maximum value of huge pages are used. Memory assumption of server applications, such as \texttt{MySQL}, \texttt{SQLite}, and \texttt{Memcached}, \texttt{Apache}, is collected by the sum of both \texttt{VmHWM} and \texttt{HugetlbPages} fields from \texttt{/proc/PID/status} file, after the corresponding client exits. 
%We always reboot server applications for each single test. 

%\end{comment}

\renewcommand{\arraystretch}{1.5}
\begin{table}[tp]

  \centering
  \fontsize{6.5}{8}\selectfont
  \caption{Memory consumption of Different Allocators in Machine B (8-node)\label{tab:memory_consumption}}
  
    \begin{tabular}{|c|c|c|c|c|c|c|c|}
    \hline
    \multirow{2}{*}{Programs}&
    \multicolumn{7}{c}{Memory Usage (MB)}\\
    \cline{2-8}
    &Linux's Default&\NM{}&TcMalloc&TcMalloc-NUMA&jemalloc&TBB&Scalloc \\ \hline
    \hline
    blackscholes&615&509&621&623&633&615&630\\ \hline
    bodytrack&37&161&45&46&570&37&1994\\ \hline
    canneal&888&879&774&757&1294&888&36149\\ \hline
    dedup&912&1236&983&1023&1389&912&8556\\ \hline
    facesim&560&500&603&601&1133&547&3056\\ \hline
    ferret&184&493&195&183&596&184&3377\\ \hline
    fluidanimate&470&392&483&484&481&470&3437\\ \hline
    raytrace&1288&1472&1092&1543&1287&1288&4398\\ \hline
    streamcluster&113&105&123&121&127&113&193\\ \hline
    swaptions&33&268&16&21&540&37&1817\\ \hline
    vips&228&536&248&269&778&227&3681\\ \hline
    x264&2859&2721&3047&3064&3719&2859&5402\\ \hline \hline  
    Aget&8&74&11&10&93&8&80 \\ \hline
    Apache&8&34&10&9&10&4&42\\ \hline
    Memcached&16&80&25&24&41&18&263\\ \hline
    Mysql&277&732&314&315&500&276& N/A \\ \hline
    Pbzip2&463&747&817&813&1121&454&4881 \\ \hline
    Pfscan&522&542&528&528&535&522&554\\ \hline
    Sqlite3&45&284&60&75&139&44&681 \\ \hline
    \hline
    Total&{\bf 9527}&{\bf 11763}&{\bf 9993}&{\bf 10510}&{\bf 14986}&{\bf 9502}&{\bf 79190}\cr\hline
    \end{tabular}
\end{table}

The memory overhead of different allocators is listed in Table~\ref{tab:memory_consumption}. In total, Intel TBB allocator has the smallest memory consumption, and the default Linux allocator is the second best one. Scalloc is the worst one in terms of memory consumption, which consumes around  $8.3\times$ more memory that that of TBB.  \NM{}'s memory consumption is around 23\% more than that of the default Linux allocator. For applications with small footprint, \NM{} may utilize up to $9.3\times$ more memory, such as \texttt{Aget}.
 
 Based on our analysis, the following reasons may lead to \NM{}'s more memory consumption. The Linux will utilize huge pages by default in machine B, if a memory area is larger than the size of a huge page (2MB). Since \NM{} utilizes \texttt{mmap} to allocate a huge chunk of virtual memory, this makes all heap memory for real objects will be allocated from huge pages. Currently, \NM{} also utilizes 1MB as the superblock for each size class, making objects of a size class that will occupy at least 1MB. Therefore, an application with many size classes will waste more memory.  
 
% \todo{Understand the data.}
    

\begin{comment}


In figure ~\ref{2node-hoard-mem}, the average normalized value of \NM{} is larger than others, but actually not too much, which is 2.3 for \NM{}, 1.9 for TcMalloc-NUMA and 1.8 for TcMalloc. It is because that proper node management is utilized in \NM{} and also in TcMalloc-NUMA, so that each node also preserves some memory not only thread locals.But we believe that this little more memory overheads are totally acceptable. It is also the same thing for figure 10, that the average value for \NM{} is little higher than others, which is 5.3. But in this 8 nodes machine, numalloc is not the worst, that Scalloc's average value is 25 and jemalloc is 9.4. One main reason that the value of \NM{} is smaller is that we use mini size bags in \NM{} which is less than the size of one page for small objects and also memories for small objects are shared per node but per cores in Scalloc.
	
\end{comment}

%\todo{Maybe we should collect the memory for 2-node machine.}

\subsection{Scalability}
\label{sec:scale}

In order to evaluate the scalability of \NM{}, we evaluate the following configurations on the Machine B: 8 threads on one node (called as 8T1N), 16 threads on one node (16T1N) and two nodes (16T2N), 32 threads on two nodes (32T4N) and 4 nodes (32T4N), 64 threads on two nodes (64T4N) and 4 nodes (64T8N), and 128 threads on 8 nodes (128T8N). Machine B is chosen since it has more cores and more nodes. \NM{}'s performance on these configurations is shown in Fig.~\ref{fig: numalloc-scalability}. For some applications, such as \texttt{facesim}, \texttt{ferret}, \texttt{fluidanimate}, \texttt{streamcluster}, \texttt{swaptions}, \NM{} scales very well. Some applications, such as \texttt{blackscholes}, \texttt{raytrace}, or \texttt{x264}, are not scalable very well. Due to the space limitation, we did not show the results of the Linux allocator, but they share the similar trend. We also observe that \NM{} typically performs better with the lower number of nodes, on a given number of threads.  This indicates that a lower number cores enables a better share of data among different threads. 

\begin{figure}[H]
    \centering
    \includegraphics[width=\textwidth]{figure/scalobility-numalloc.pdf}
    \caption{Scalability of \NM{} with different configurations.\label{fig: numalloc-scalability}}
\end{figure}

\begin{figure}[!h]
    \centering
    \includegraphics[width=\textwidth]{figure/scalability-pthread.pdf}
    \caption{Normalized performance of Linux's default allocator without binding for PARSEC benchmarks in Machine B}
    \label{pthread-scalibity}
\end{figure}

 In order to understand the scalability of \NM{} when comparing to other allocators, we utilize synthetic applications. The reason of utilizing synthetic applications is that they were designed to be scalable~\cite{Scalloc}. Therefore, we could eliminate the scalability issue caused by applications. Since other allocators cannot specify the configuration, we only evaluate the scalability with different number of threads. For \NM{}, we maximize the number of threads on each node. For instance, the result of 32 threads will use 2 node, since each node has 16 cores. \todo{XZ: please utilizes the same method as scalloc, and lists all data like Figure 7 to Figure 9 of Scalloc paper. The corresponding data is shown between Fig.~\ref{} and Fig.~\ref{}.} 


\begin{figure}[!ht]
    \centering
    \includegraphics[width=\textwidth]{figure/hoard-perf.pdf}
    \caption{Normalized runtime with different allocators for Hoard benchmarks}
    \label{hoard-perf}
\end{figure}

\begin{figure}[!ht]
    \centering
    \includegraphics[width=\textwidth]{numalloc/paper/figure/sythentic-scalobility.pdf}
    \caption{Speedup with respect to the default Linux allocator}
    \label{sythentic-scalility}
\end{figure}

For synthetic applications, we are using four benchmarks from Hoard~\cite{Hoard}, including \texttt{threadtest}, \texttt{larson}~\cite{Larson}, \texttt{cache-scratch} and \texttt{cache-slash}, which is also employed by existing work~\cite{Scalloc}. 

\texttt{threadtest} is an application that performs a large number of allocations and deallocations with a specified number of threads. Also, it allows to specify how much work to be done between each allocation and deallocation. For \texttt{threadtest}, we use 100 iterations, 1,280,000 allocations, 0 work, and 64-byte objects (for the allocation).  Basically, this benchmark will stressfully test the performance overhead of allocation and deallocation. For \texttt{threadtest}, the reason why \NM{} achieves much better performance than other allocators can be caused by the following reasons: first, \NM{} imposes very minimal system call overhead, since it allocates a large chunk of memory at one time. By comparison, \texttt{TcMalloc} will invoke \texttt{mmap} for every page. Second, \NM{} imposes little synchronization overhead, since every thread will has its own heap and it only imposes some when getting objects from the shared bag. But \NM{} obtains a number of objects at a time, at the page level, which significant reduce the possibility of contention. 

The \texttt{larson} is to simulate a multithreaded server that could respond to requests from different clients. Each thread   will receive a random number of objects in the beginning, perform a random number of allocation and deallocations to simulate the handler for processing requests, and then pass objects to the next thread. We test \texttt{larson} for 10 seconds with 1,000 objects for 10,000 iterations, where each allocation is between 7 bytes and 2048 bytes. 


\texttt{cache-scratch} tests passive false sharing, and \texttt{cache-thrash} tests active false sharing. False sharing occurs when multiple threads are concurrently accessing different words in the same cache line. Since allocators cannot control false sharing inside one object, false sharing of allocators mainly focuses on whether objects are allocated from the same cache line or not. It further includes two types of false sharing.   Passive false sharing is introduced upon deallocations, where the reallocation of freed objects introduces the false sharing. In contrast, active false sharing can be introduced during the first allocation of objects, where each thread is not responsible for allocating continuous objects.   For false sharing tests, we use 100,000 inner-loop, and 100,000 iterations with 8-byte objects. The performance data can be seen in Fig.~\ref{hoard-perf}, where the normalized runtime is shown. \NM{} will not introduce active false sharing, since each thread will get a page. That is the reason why it has a good performance as the Linux allocator for \texttt{cache-thrash}. Although \NM{} might introduce passive false sharing in theory, due to its per-thread cache design. However, \NM{} avoids remote allocations, since every thread only places a freed object into its local cache when it is originated from the same node.  Other allocators do not have such mechanisms.   


When the number of threads is equal to the number of cores, \NM{} has the best overall performance for both two machines, and for almost all single applications. On machine 1, \NM{} is around 2 $\times$ faster than the default allocator. \NM{} is around 3 $\times$ faster than the default one (the next best one) on machine 2.  

 
 
\begin{comment}

\begin{figure}[!ht]
    \centering
    \includegraphics[width=\textwidth]{figure/scalobility-pthread.pdf}
    \caption{Normalized performance of Linux's default allocator without binding for PARSEC benchmarks in Machine B}
    \label{pthread-scalibity}
\end{figure}
We will evaluate the scalability on 8threads, 16threads, 32 threads, 64 threads and 128 threads. 
(one node, two node, four nodes, and 8 nodes). 
	
\end{comment}


\subsection{Design Choices}
\label{sec:design}

This section further confirms multiple design choices of \NM{}. Some mechanisms are very basic, such as node-aware memory allocation, which is mandatory for NUMA-aware memory allocators, and cannot be evaluated separately. Therefore, they are not evaluated here. Some design choices may only help on one or two applications, such as efficient objects migration. 

\subsubsection{Interleaved Heap} 
\label{sec:interleavedheap}

\begin{figure}[H]
    \centering
    \includegraphics[width=\textwidth]{figure/interleavedheap.pdf}
    \caption{Normalized runtime with and without interleaved heap.\label{fig:interleavedheap}}  
\end{figure}

We first evaluate the performance difference when the support of interleaved heap is enabled or not. As described in Section~\ref{sec:performance}, some applications, especially those ones having a large portion of time in the serial phase, may not have good performance with the interleaved heap support. The interleaved heap will allocate the memory from all nodes interleavedly, instead of from the local node (based on the default first-touch policy). The interleaved heap could be utilized to avoid load imbalance issue for shared objets. 

However, there are two issues for the interleaved heap. First, the allocator may not know whether an object is shared or not at the first time. Therefore, all objects that are allocated in the main heap (before creating any child thread) will be treated as the shared heap. However, a private object, if it is allocated interleavedly in multiple nodes, may introduce unnecessary overhead due to remote accesses. Second, some applications are spending too much time in the serial phase, where the interleaved heap cannot benefit the performance for the serial phase. 

The performance data with and without interleaved heap is shown in Fig.~\ref{fig:interleavedheap}. Note that we have evaluated all real applications listed in Section~\ref{sec:performance}. The applications that have no or little performance impact by the interleaved heap are omitted in this figure. From Fig.~\ref{fig:interleavedheap}, we have the following conclusion: the interleaved heap will benefit (or at least no harmful impact) the performance for most applications, except applications with a large portion of serial phase (e.g., \texttt{canneal} and \texttt{raytrace}). Some applications, such as \texttt{fluidanimate}, will have the performance speedup of $3.5\times$ with the interleaved heap. Therefore, the interleaved heap can be enabled by default, unless programmers know that it will not benefit the performance. A simple metrics is to use the portion of its serial phase. 


%figure ~\ref{parsec-no-interleaved-perf} we show some performance results of some applications that got significant different values after we shut down interleaved heap for \NM{}. We can see that for some applicatios with less data sharing between threads like ratrace and canneal, \NM{} could got significant improvements due to its low overheads and proper memory management. But for some other applications with intensive memory operations and sharing like fluidanimate, shutting down interleaved heap could hurt performance, since interleaved heap could help to distributed resource contention evenly over multi-nodes and then got low overheads.

\subsubsection{Thread Binding}
\label{sec: threadbinding}

We will only run it on the 8-node machine. \todo{confirm the results of thread binding.} 

\begin{figure}[!h]
    \centering
    \includegraphics[width=\textwidth]{figure/WO-pthread-binding.pdf}
    \caption{Normalized performance of Linux's default allocator with the binding in Machine B}
    \label{binding-pthread-scalibity}
\end{figure}

\subsubsection{Huge Page Support} 
Since Machine B will utilize huge pages by default, we will evaluate the performance impact of huge page support on Machine A (2-node machine). We only utilize PARSEC applications for this evaluation. 

\begin{figure}[!h]
    \centering
    \includegraphics[width=\textwidth]{figure/hugepage.pdf}
    \caption{Normalized runtime with and without huge page support on Machine A (2-node).}
    \label{fig:hugepage}
\end{figure}

The results are shown in Fig.~\ref{fig:hugepage}. When integrating with huge page support, \NM{} achieves a significantly better performance for \texttt{dedup}, where the performance difference is around 15\%. On average,  the huge page support improves the performance of all evaluated applications about 2.5\%. This clearly indicates that it is beneficial to have huge page support integrated with the memory allocator.  



\section{Limitation}
\label{sec:limit}

This section describes some limitations of \NM{}. 

The first limitation of \NM{} is that it consumes more memory than some popular allocators. Based on our analysis, the memory consumption is caused by \NM{}'s bag mechanism. Currently, \NM{} employs one MB as a bag, which indicates that objects of each size will occupy at least of 1 MB, when transparent huge page support is enabled. This design achieves a fast lookup on the metadata, but will utilize more memory unfortunately. We will investigate whether reducing the size of a bag could help reduce the memory consumption in the future.

The second limitation is that \NM{} may introduce some reliability issue by preallocating a huge chunk of memory from the OS in the beginning. A program that will crash in other allocators may not crash, if the address is landed on the pre-allocated range. Instead, \NM{} aims to achieve the high performance over the reliability. 
\section{Related Work}

\paragraph{General Purpose Allocators:}

\paragraph{NUMA-aware Allocators:} 

Ogasawara focuses on finding the preferred node location for JAVA objects during the garbage collection and memory allocations~\cite{Ogasawara:2009:NMM:1640089.1640117}, via thread stack, synchronization information, and object reference graph. The proposed method is not suitable for C/C++ applications, since the objects were tighten to physical pages. 

~\cite{wagle2015numa} focuses on the specific scenario, which is in-memory databases. 

~\cite{kim2013node}

~\cite{tcmallocnew} utilizes two mechanisms to support the NUMA architecture based on TcMalloc. First, it adds additional node-based freelists and free spans to store freed objects and pages belonging to the same node. Second, it also invokes the \texttt{mbind} system call to bind physical memory allocations to the node that the current thread is running on.  However, it does not support huge pages and the special allocations from the main thread, invokes too many \texttt{mbind} system calls, and does not handle the metadata's locality. 



%However, there are multiple issues of this approach. It always assumes that the physical memory will be always allocated in the same node as the thread who requests the memory. It is not true due to the load-balance inside the OS, and some threads may allocate some objects that are exclusively used by other threads. Second, it did not handle the metadata of different nodes. Third, it does not bind threads to different nodes, which may cause large amount of remote accesses when a thread is scheduled to a different core. 



~\cite{Majo:2015:LPC:2688500.2688509} proposes to set task-to-thread affinity, and pin threads to specific cores to achieve a better performance. It is not a memory management policy, and mainly talks about the detail implementation of task binding over different levels of the TBB library. 

NumaGiC reduces remote accesses in garbage collection phases with a mostly-distributed design that each GC thread will mostly collect memory references locally, and utilize a work-stealing mode only when no local references are available~\cite{NumaGiC}.  


~\cite{diener2015automatic} proposes to combine the task management and memory management to achieve the better performance. For task mapping, it proposes to place tasks that shared the data onto the cores that share the cache, in order to reduce the cache misses and the communication overhead. For memory management, it utilizes the page faults to analyze the memory access behavior, and then migrate pages between nodes to improve the performance. Basically, this is still a proactive approach, which cannot achieve the optimal performance promised by the hardware. Also, it requires the changes of the underlying OS, and will impose some overhead of understanding the communicating tasks and memory access behavior via analyzing page faults.  

 
~\cite{1419934}
Kaminski et al. proposes to make TCMalloc NUMA-aware~\cite{tcmallocnew}, with the very minimum effort. The idea is to maximize the local memory allocation with the node-based free list and page heap. However, this mechanism assumes that the memory deallocation from the current node will be always allocated from the local node. This is unfortunately not true for many cases, such as a producer-consumer model. Also, this allocator does not take advantage of 

~\cite{Majo:2011:MMN:1993478.1993481} proposes to consider both data locality and cache contention, and combine memory management with task scheduling to achieve better performance. Mostly, it is focuses on the task scheduling. 
 
We describe two scheduling algorithms: maximum-local, which optimizes for maximum data locality, and its extension, N-MASS, which reduces data locality to avoid the performance degradation caused by cache contention. N-MASS is fine-tuned to support memory management on NUMA-multicores and improves performance up to 32\%, and 7\% on average, over the default setup in current Linux implementations.

%~\cite{memarzia2019toward} investigates the performance impact of utilizing different memory allocators, different task assignment, different memory allocation policies, and different OS configuration. It showed that some specific allocator could improve the performance up to $20\times$ when using the interleaved memory allocation policy and modify the OS configuration. 

% http://memkind.github.io/memkind/ 
% Use this to compare the performance difference
% However, this will require the change of applications to use their library. 
~\cite{cantalupo2015memkind} proposes a new library that allows users to manage their memory in fine granularity by combining with multiple existing system calls. However, they are not targeting for a general purpose allocator, since it requires programmers to manage the memory explicitly. Unfortunately, this place unacceptable burden to programmers. More importantly, the explicit management based on one existing topology may not work well for the hardware with a different topology. 



\paragraph{Other NUMA-related Systems:} 

Memory system performance in a numa multicore multiprocessor

~\cite{Majo:2015:LPC:2688500.2688509} proposes TBB-NUMA, a system that mainly focuses on task scheduling on NUMA architecture to achieve better performance. \NM{} employs a similar mechanism as TBB-NUMA to manage threads explicitly, but without relying on the human hints. \NM{} also deals with the memory allocation that is not presented in TBB-NUMA. 

~\cite{6704666} shows that a set of simple algorithmic changes coupled with commonly available OS functionality suffice to eliminate data sharing and to regularize the memory access patterns for a subset of the PARSEC parallel benchmarks. These simple source-level changes result in performance improvements of up to 3.1X, but more importantly, they lead to a fairer and more accurate performance evaluation on NUMA-multicore systems. In the default
configuration, OSs (such as Linux) tend to change the thread-to-core mapping during the execution of programs, which result in large performance variations.

~\cite{Bui:2019:EPV:3302424.3303960} talks about the virtualization of NUMA.

~\cite{jemalloc} introduces a round-robin fashion for arena allocation, and all locks and buffers will be local to the associated arena. Similarly, \NM{} also takes the similar approach. 

Scalloc utilizes the same-sized spans in order to encourage memory reuses~\cite{Scalloc}, which is also the same for \NM{}. Salloc has another two contributions, a global backend developed by concurrent data structures, and a constant-time front end that returns the spans to the backend. 

Bolosky et. al. propose a simple mechanism to improve the performance by replicating read-only pages to multiple processors, moving pages to the processor that written them, and placing pages to the global memory if they are written by multiple processes ~\cite{Bolosky:1989:SBE:74850.74854}. bl


\section{Conclusion}
\label{sec:conclusion}

\NM{} is a memory allocator that is specially designed for the NUMA architecture. Applications can be linked to \NM{} directly, without the change of code and the recompilation. \NM{} is different from existing memory allocators, as it is the first binding-based allocator. 
On top of it, it further proposes origin-aware memory management and incremental sharing to improve the locality and exploit huge pages. 
% It further proposes threads-shared incremental allocation and origin-aware memory management to improve the locality. 
Based on our extensive evaluation, \NM{} achieves a significantly better performance than other popular allocators on the NUMA architecture, which is running \NEW{15.7\%} faster (and up to $6.8\times$ faster) than the second-best allocator.
%\NM{} is available at https://github.com/XXX. 
\bibliographystyle{abbrv}
\bibliography{ref,emery,tongping}


\begin{IEEEbiography}[{\includegraphics[width=1in,height=1.25in,clip,keepaspectratio]{fig/tongping}}]{Tongping Liu}
 received the PhD and MS degree from University of Massachusetts Amherst. He is currently an assistant professor at the Department of Electrical and Computer Engineering in the University of Massachusetts Amherst. Prior to that, he worked at University of Texas at San Antonio between 2014 and 2019. His research interests include the performance, reliability and security of software systems. 
\end{IEEEbiography}

\begin{IEEEbiography}[{\includegraphics[width=1in,height=1.25in,clip,keepaspectratio]{fig/xin.jpeg}}]{Xin Zhao}
Xin Zhao received the BS degree from Harbin Institute of Technology, China. He is currently a PhD student at the University of Massachusetts Amherst.
\end{IEEEbiography}

\begin{IEEEbiography}[{\includegraphics[width=1in,height=1.25in,clip,keepaspectratio]{fig/weiwang}}]{Wei Wang}
Wei  Wang holds  a  Ph.D.  in  computer  science from  University  of  Virginia  in  2015.  He  is  currently  an  Assistant  Professor  at  the  Computer Science Department of the University of Texas at  San  Antonio.  His  research  interests  include system software, computer architecture, cloud computing, and software  engineering. He is a member of the IEEE.
\end{IEEEbiography}


\begin{IEEEbiography}[{\includegraphics[width=1in,height=1.25in,clip,keepaspectratio]{fig/bo-wu}}]{Bo Wu}
Bo Wu is an associate professor at the Colorado School of Mines, and has a PhD in Computer Science from the College of William and Mary. His research lies in the broad field of compilers and programming systems, with an emphasis on program optimizations for heterogeneous computing and emerging architectures. 
%He was the recipient of IBM CAS fellowship from 2011 to 2013 and Stephen K. Park award for graduate research from the College of William and Mary.
\end{IEEEbiography}


\begin{IEEEbiography}[{\includegraphics[width=1in,height=1.25in,clip,keepaspectratio]{fig/Sandip_Kundu}}]{Sandip Kundu}
Sandip Kundu is a professor of electrical and computer engineering at the University of Massachusetts Amherst. Previously, he was a principal engineer at Intel Corporation and a research staff member at IBM Corporation. He has published more than 200 papers in VLSI design and CAD, holds 12 patents, and has co-authored multiple books. 
%He served as the technical program chair of ICCD in 2000, co-program chair of ATS in2011, ISVLSI in 2012, and 2014, DFT in 2014. 
He is a fellow of the IEEE and has been a distinguished visitor of the IEEE Computer Society.
\end{IEEEbiography}



\end{document}
